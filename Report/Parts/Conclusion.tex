{Conclusion}
\section{Discussion}
In this project, we have investigated the possibilities and challenges of setting up a DNS forwarding server (BIND). This has been done from the perspective of an imaginary real-life case. 

A gradual approach, where we first found a fitting DNS name-server and then configured the BIND server to forward DNS requests to this server. 

We have achieved results within our expectation; caching reduces the DNS lookup time in subsequent queries. 

\section{Conclusion}
To conclude, we achieved the desired results. 

We have configured a BIND server to listen on localhost and setup a Linux box to resolve DNS queries through the localhost interface. 
This was done fairly easily and it is definitely within the realm of possibilities to setup a DNS resolver for anyone, managing any type of network. 

Google Namebench was used to determine an optimal DNS name-server. Using this, more optimal server, gave a definite boost of the DNS lookup time.

We have learned that DNS is one of the key aspects of how the internet works. And we have thereby gained a lot more insight in how one of the most important technologies of today works in practice.

\section{Perspectives}
DNS is an important technology and doesn't seem to be replaced by anything soon. It would be great if everyone sat up their own BIND caching server to improve lookup speed and to reduce the load of the global DNS servers.
