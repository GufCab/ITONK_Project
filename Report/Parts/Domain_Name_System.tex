\chapter{Domain Name System}
\section{Overview}
This section will describe the technological fundamentals of DNS, as well as related technologies like IP addressing. The technologies will be described along with example software used to display different parts of the theory.

The section will also contain a description of the BIND DNS software, along with installation and user guide. 

This section will not focus on the chosen real-life case. This is a general description of the technology.
\section{IP Addressing}
\subsection{IP overview}
To access any resource on the internet (or LAN), be it a website or a specific device, a location is needed. When dealing with internet-related technologies, this location is specified by an \textit{IP address}. 

On any given network, an IP address should be unique to a specific device/server. However, local network routers generally have a private block of IP addresses for devices on that specific sub-net and a separate IP for the outside world (access to sub-net devices are forwarded to the specific devices by the router).

The router then acts as the \textit{gateway} to the outside world and should be accessible by an IP within the range of the sub-net.
To check the IP address of a device, the tool ifconfig (Linux) or ipconfig (windows) can be used 

\begin{center}
	\includegraphics[width=\textwidth]{ifconfig.png}
	\captionof{figure}{ifconfig command from Linux machine}
\end{center}

If the tool \textit{nm-tool} is run (on a Linux machine) the output will look like this:

\begin{center}
	\includegraphics[trim = 0px 100px 400px 47px, clip, width=\textwidth]{nm-tool.png}
	\captionof{figure}{nm-tool command from Linux machine}
\end{center}

Where 

\textbf{Address} 
Denotes the machines' IP on the local sub-net. Other devices on the sub-net will use this IP on to communicate with the device.

\textbf{Prefix} 
Denotes the subnet - mask of the network. This shows how many individual addresses are available on the sub-net. An IP address is 32 bits total and in the \textit{xx.xx.xx.xx/yy} format, \textit{yy} denotes the amount of bits reserved for the address of the \textit{subnet} and not the address of the individual devices on the net. Thus, a subnet with fewer bits reserved for the subnet address (say, 16 bits) can contain more individual addresses. This can be utilized in a larger sub-net, containing smaller sub-nets with the individual devices and thus fanning out to the individual addresses. 

\textbf{Gateway}
As explained above, most devices are connected to a sub-net, where their address only has to be individual on the sub-net. The \textit{gateway} then, is the point of access to the outside world. This could be the main router, of the subnet. All access to the internet goes through this device and the Gateway field, obtained when running nm-tools, describes the address of this device. 

\textbf{DNS}
Tells us the address of the first DNS server, we will query for an address of a resource on the internet. When a user types the URL of a website, e.g. www.google.com, this IP address is queried for the actual address of the URL, this is explained in detail further into this report.


\section{Name Resolution and forwarding}
%Hvad er formålet med name resolution? Hvad gør det? Hvad får vi ud af det?
\subsection{Iterative}
\subsection{Recursive}
\subsection{Caching}
\subsection{Security [DNSSEC]}
\section{BIND}
\subsection{Downloading}
\subsection{Configuration}
\subsection{Basic Use}

%This chapter should contain an in-depth description of the technology,
%i.e. DNS, DDS, or RMI. You should at least address
%\begin{itemize}
%\item The purpose of the technology
%\item Technology alternatives
%\item Downloading, installing, configuring, and employing the technology
%\end{itemize}
