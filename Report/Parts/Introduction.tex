\chapter{Introduction}
The introduction of the Internet heralded an entirely new way of communication between humans. At an extreme pace, we are moving towards a world where every human being has access to an internet connection, to a point where several humanitarians and philanthropists perceive internet access as a big part of the solution towards poverty and human rights abuses.

But what binds this enormous web of people together? How do we 'find' each other on the World Wide Web?

This report will give an in-depth description of the Domain Name System (DNS), the technology responsible for translating \textit{names} to \textit{addresses}, making it possible to type in a human-readable domain name into a browser and have it translated to the specific address of a web server, anywhere in the world.

DNS servers act as the way-signs of the internet, directing the end user to the address of a desired server/domain. Thus, the end user really only needs to know the address of a DNS server, that can point him further in the right direction.

The report will focus on the technology from the case of a kindergarten teacher, Uffe, in need of setting up better DNS forwarding. 

This is relevant because it shows a real life application of setting up a local DNS server with the BIND software. The actual tangible results of caching and forwarding to the optimal DNS server are shown in the results section.

The report is structured thus
\begin{enumerate}
\item Description of DNS
\item The role of DNS in the real-life case
\item Prototyping with BIND
\item Perspective and conclusion
\end{enumerate}

As to first give a description of the basic theory before diving into a real-life application. To get an idea of the role of DNS, a section describing the role of the technology in the scope of the case is included before diving into the nitty gritty details of setting up a DNS server.